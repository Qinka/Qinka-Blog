\documentclass{ctexart}
\usepackage{listings,amsmath,amstext}

\author{Qinka}
\title{灰度预测}

\begin{document}
  \maketitle
  %\section{Summary}
这篇 POST 是关于使用 $GM(1,1)$ 灰色预测模型预测的代码。

这篇文章是与灰色预测有关的内容,用于备用。代码一共四个文件
\begin{enumerate}
    \item \href{/r/source/matlab/basicgrey.m}{basicgrey.m}
    \item \href{/r/source/matlab/judge.m}{judge.m}
    \item \href{/r/source/matlab/lcheck.m}{lcheck.m}
    \item \href{/r/source/matlab/relations.m}{relations.m}
\end{enumerate}
代码非原创。
  \section{使用说明}
  \label{sec:note}
  主要用于预测的是函数 \lstinline|basicgrey(X,m)|
  其中 $X$ 是已有的数据组成的矩阵, $m$ 是所要预测的数量。
  $X$ 所需要的数据是不少于 3 个。这个是用于短期预测。
  
  返回的值分别是
  \begin{description}
      \item[var] 预测的结果
      \item[ago] 预测后累加值
      \item[alpha] 预测的 $\alpha$ 与 $\beta$
      \item[error] 预测的残差
      \item[P] 计算的小残差的概率
      \item[c] 后验差比值
      \item[rela] 关联度
      \item[t] 对于预测的评价
  \end{description}
    
  预测评价是由\lstinline|judge| 进行的。
  返回的是含有两个元素的矩阵 $t$。
  $t_1$ 表示的是对于关联度检验的结果,
  其中 1 表示:“依据经验关联度检验结果为满意。”,
  反之表示不满意。
  $t_2$ 表示的是根据后检验差结果的显示,
  这个模型的评价,如下表格。
  \begin{center}
  \begin{tabular}{c|c}
       \hline
       $t_2$取值 & 描述 \\
       \hline
       4 & 模型评价为“优” \\ 
       3 & 模型评价为“合格” \\ 
       2 & 模型评价为“勉强合格” \\ 
       1 & 模型评价为“不合格” \\ 
       \hline 
  \end{tabular} 
  \end{center}
\end{document}