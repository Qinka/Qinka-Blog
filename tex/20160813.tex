\documentclass{ctexart}
\usepackage{listings,amsmath,amstext}
\usepackage{hyperref}
\usepackage{appendix}
\renewcommand\appendixname{附录}
\renewcommand\appendixpagename{附录}
\appendixtitleon

\author{Qinka}
\title{2016-08-13 题目内容}
\def\pandoc{1}
\begin{document}
    \ifx\pandoc \undefined
    
\section*{Summary}
\label{sec:sum}
这个是第零道题目的相关资料。其中较为重要的部分有如下
\ifdefined\pandoc
\begin{enumerate}
    \item \href{#code:cc}{层次分析法中的代码}
    \item \href{#sec:chi2}{卡方分布的概率密度函数}
    \item \href{#sec:sumtaxi}{利用 MC(MC) 方法 经行模拟出租车应答}
\end{enumerate}
\pandoc
\else
\begin{description}
    \item [Par~\ref{code:cc}]  层次分析法中的代码
    \item [Section~\ref{sec:chi2}] 卡方分布的概率密度函数
    \item [Section~\ref{sec:sumtaxi}] 利用 MC(MC) 方法 经行模拟出租车应答
\end{description}
\fi
附带的文件见附录一。
\endinput
    \fi
    $$\left\{ \begin{array}{lll}
    x & = & \{-a + \frac{\varphi}{2},-a + \frac{3\varphi}{2},\dots,a - \frac{\varphi}{2}\}  \\
    y & = & 2l\cos \theta-\frac{d}{2}\\
    & = & 2l\cos \theta - \frac{l}{2}\frac{\sin(\frac{\varphi}{2}-\theta)}{\sin(\frac{\varphi}{2})}\\
    z & = & w \cos \varphi 
    \end{array}$$
    
    $$\varphi ; \pi \rightarrow \frac{pi}{2} \rightarrow \arcsin\frac{H}{w}$$
    
    $$\frac{2l}{\sin\varphi} = \frac{w}{\sin\theta}$$
    
    $$H=2L\sin\theta$$
    
    $$\left\{\begin{array}{lllll}
    (y+d)^2+x^2 &=& a^2&,&y\leq -d\\
    (y-d)^2+x^2 &=& a^2&,&y\geq d \\
    \end{array}$$
    
    $$d:l\rightarrow 0$$
    
    $$\left\{\begin{array}{lllll}
    (y+w+d)^2+x^2 &=& a^2&,&y\geq -(w+d)\\
    (y-w-d)^2+x^2 &=& a^2&,&y\leq w+d \\
    \end{array}$$
    
    $$\left\{\begin{array}{lll}
    x &=& \pm a\\
    y &=& \pm(L-l+d)
    \end{array}$$
    
    $$\frac{2d}{\sin(\frac{\varphi}{2}-\theta)}=\frac{2l}{\sin\frac{\varphi}{2}}$$
    
    \section{第一问}
    \label{sec:q1}
    北京(A),南京(B),太原(C) 三类城市的2013-2014年度的数据
    \subsection{出租车}
    $x_1$ 10公里打车费用站人均收入比例
    $$x_1=(0.08 ,0.08,0.09)$$
    单位是 \%
    
    $x_2$ 驾龄需求
    $$x_2=(3,1,1)$$
    
    $x_3$ 证件需求
    $$x_3=(A,A,A)$$
    
    $x_4$ 数量 (析构)
    
    $x_5$ 覆盖规模
    $$x_5=(100,100,100)$$
    单位是 \%
    
    $x_6$ 车型
    
    $x_7$ 从业人员(占比)
    $$x_7=(0.25,0.13,0.29)$$
    单位是 \%
    
    $x_8$ 经营方式
    $$x_8=\left(\begin{array}{ccc}
    0.0 & 0.4 & 1.9 \\
    5.7 & 18.6 & 16.2\\
    0.0 & 0.0 & 1.2\\
    94.3 & 81.0 & 80.7
    \end{array}\right)$$
    其中 每行分别代表 个体经营,挂靠经营,合作经营,公司经营,单位为 \%。
    
    $x_8$ 运营方式
    $$x_8=\left(\begin{array}{ccc}
    51.0 & 6.3 & 12.8 \\
    41.0 & 64.3 & 77.3 \\
    8.1 & 29.4 & 9.9
    \end{array}\right)$$
    其中每一行,分别代表 三班倒,两班倒,与一班倒,单位为 \%。
    
    $x_9$ 车辆保有量
    $$x_9 =(26.5, 15.0 ,25.2)$$
    单位是 \%
    
    $x_{10}$ 排量中位数
    $$x_{10} = (1.7,1.6,1.6)$$
    
    $x_{11}$ 人群
    $$x_{11} = \left(\begin{array}{ccc}
    28.6 & 23.5 & 24.5\\
    28.6 & 29.4 & 22.4\\
    14.3 & 11.8 & 16.3\\
    14.3 & 17.6 & 24.5\\
    14.3 & 17.6 & 12.2
    \end{array}\right)$$
    其中各行依次为工薪阶层,白领,学生,一般居民,其他
    单位是 \%
    
    $x_{12}$打车难易度
    $$x_{12}=\left(\begin{array}{ccc}
    0.0 & 0.0 & 5.6\\
    50.0 & 33.3 & 44.4\\
    50.0 & 50.0 & 38.9\\
    0 & 16.7 & 11.1
    \end{array}\right)$$
    其中,各行依次代表 很难,难,一般,及不难,单位为\%。
    \subsection{网约车-滴滴}
    驾龄:一年以上\footnote{专车3年}
    
    证件:身份证,驾驶证,行驶证
    
    覆盖规模
    $$x_5=(100,100,88.6)$$
    单位 \%。
    
    从业人员(占比)
    $$x_7=(,,)$$
    
    运行方式
    $$x_8=\left(\begin{array}{ccc}
    0 & 0 & 0\\
    0 & 0 & 0\\
    100 & 100 & 100
    \end{array}\right)$$
    
    车辆保有量
    
    排量
    
    人群
    
    \section{第二问,层次分析}
    \label{sec:q2}
    第二问层次分析所用到的矩阵
    $$A=\left(\begin{array}{ccc}
    1            &  2            &  4      \\
    \frac{1}{2}  &  1            &  2      \\
    \frac{1}{4}  &  \frac{1}{2}  &  1     
    \end{array}\right)$$
    $$B_1=\left(\begin{array}{ccc}
    1            &  \frac{1}{3}  &  5      \\
    3            &  1            &  7      \\
    \frac{1}{5}  & \frac{1}{7}   &  1      
    \end{array}\right)$$
    $$B_2=\left(\begin{array}{ccc}
    1           & 7           & 9 \\
    \frac{1}{7} & 1           & 3 \\
    \frac{1}{9} & \frac{1}{3} & 1
    \end{array}\right)$$
    $$B_3=\left(\begin{array}{ccc}
    1           & 3 & \frac{1}{3} \\
    \frac{1}{3} & 1 & \frac{1}{5} \\
    3           & 5 & 1
    \end{array}\right)$$
    
    \paragraph{代码}
    \label{code:cc}
    编写matlab函数。
    \begin{lstlisting}[language=matlab]
function [ W,v,f,CR ] = classclass( A )
%classclass 函数 classclass 用于计算层次分析法中的 W
% 输入  A 评价矩阵
% 返回  W 是计算的 W
%       v 是 矩阵 A 最大特征值对应的特征向量
%       f 是一致性检验的结果(真值)
    n = max(size(A));
    sumk = sum(A);
    W = zeros(n,1);
    for i = 1:n
        for j=1:n
            W(i) = W(i) + A(i,j)/sumk(j);
        end
        W(i) = W(i)/n;
    end
    [V ,D] = eig(A);
    CI = (max(max(D))-n)/(n-1);
    RI = [0 0 0.58 0.90 1.12 1.24 1.32 1.41 1.45 1.49 1.51 1.54 1.56 1.58 1.59];
    CR = CI/RI(n);
    f = CR < 0.10;
    v = V(:,1);
    v = v/(sum(v));
end
    \end{lstlisting}
    分别计算 矩阵 $A$,$B_1$,$B_2$,$B_3$ 对应的??
    得到
    \begin{align*}
    V_A = (0.5714,0.2857,0.1429)^T \\
    V_{B_1} = (0.2790,0.6491,0.0719)^T \\
    V_{B_2} = (0.7854,0.1488,0.0658)^T \\
    V_{B_3} = (0.2583,0.1047,0.6370)^T\\
    \end{align*}
    
    \section{某问}
    \label{sec:qq}
    \subsection{太原出租车价格}
    \label{ssec:taxipr:ty}
    \begin{tabular}{|c|c|c|c|}
        \hline \multicolumn{2}{|c|}{收费项目} & 计价单位 & 收费标准(元)\\ 
        \hline 日间 & 起步价 & 3公里 & 8 \\ 
   \cline{2-4}  & 基本单价 & 公里 & 1.6 \\ 
        \hline 夜间 & 起步价 & 3公里 & 8 \\ 
   \cline{2-4}  & 基本单价 & 公里 & 1.8 \\ 
        \hline
    \end{tabular}
    \paragraph{备注}
    \begin{enumerate}
        \item 夜间时间: 21:00-5:00
        \item 运营重停车等待计时,累计5公里以1公里记价
        \item 单程空帖,10公里以上基本租价加收$50\%$
        \item 油运联动:
        当93号汽油的价格在$[8.5,9.5)$元/升 之间,加收 1元燃油附加费\footnote{单位是1元/趟·次,下同},
        在$[9.5,10.5)$元/升 之间,再加收一元燃油附加费。
        油价下降时,逆向操作。
        油价超出$[7,10.5)$这个区间时,适时启动客运出租车
    \end{enumerate}
    
    \section{卡方分布}
    \label{sec:chi2}
    生成的数据满足 $Q\sim\chi^2_n$,其中卡方分布的概率密度函数
    $$f(x)=\begin{cases}
    \frac{x^{\frac{n}{2}-1}e^{-\frac{x}{2}}}{2^{\frac{n}{2}}\Gamma(\frac{n}{2})} ,& x \geq 0\\
    0,& otherwise
    \end{cases}$$
    
    \section{车辆应答模拟}
    \label{sec:sumtaxi}
    使用 Literate Haskell 方式(语言)编写,的 有关于车辆应答模拟的 MCMC 代码\footnote{参见附录}。、
    代码及说明参见 \href{/b/sumo/T/20160813/code/sumtaxi.lhs}{车辆应答模拟代码}。
    
    \begin{appendices}
        \section{附带文件}
        \label{apd:exfile}
        \begin{enumerate}
            \item
            \href{/r/pdf/20160813.pdf}{本文的pdf版本}
            \item
            \href{/r/doc/sumtaxi.docx}{车辆应答模拟代码的 Word 文档}
            \item
            \href{/r/lhs/sumTaxi.lhs}{车辆应答模拟代码}
        \end{enumerate}
        \section{大量运行车辆应答模拟的POSIX-Shell 脚本}
        \begin{lstlisting}[language=bash]
#!/bin/bash
rm -f $2
for V in $(seq 1 1 $1)
do
./sumTaxi $3 $4 $5 $6 $7 $8 $9 ${10} >> $2
done
echo '>'
cat $2 |sed  "s/,/\ /g"|  awk '{print ($1 > $2)}' | grep -c 1
echo '>='
cat $2 |sed  "s/,/\ /g"|  awk '{print ($1 >= $2)}' | grep -c 1
echo didi time
cat $2 |sed  "s/,/\ /g"|  awk '{print $1}' | awk '{sum+=$1} END {print "avg=",sum/NR}'
echo taxi time
cat $2 |sed  "s/,/\ /g"|  awk '{print $2}' | awk '{sum+=$1} END {print "avg=",sum/NR}'
        \end{lstlisting}
        其中参数分别是 运行规模,输出文件,出租车速,网约车速,乘客可视半径,网约车辆数目,
        出租车车辆数目,模拟区域大小,上限时间倍数,与单次扫描时间。
        当脚本运行完之后会依次输出 出租车比网约车好的结果个数,出租车不差于网约车的个数,
        网约车的响应时间,与出租车的平均响应时间。
        \section{附带公式}
        这部分是附带的一部分的公式,出现在论文中的。
        $$d_1=\eta_1 e_1+ \xi_1-c_1$$
        $$d_2=\eta_2 e_2+ \xi_2-c_2$$
        $$d_{ij\text{基础}}=0$$
        $$d_{11}=d_1+m_1=d_1+0.12$$
        $$d_{12}=0+m_2=0.14$$
        $$d_{21}=0+m_1=0.12$$
        $$d_{22}=d_2+m_2=d_2+0.14$$
        $$D=\left(\begin{array}{cc}
        d_{11} & d_{12} \\
        d_{21} & d_{22}
        \end{array}\right)$$
        $$D^1=\left(\begin{array}{rr}
        175.3200 &   0.1400\\
        0.1200 & 242.7400
        \end{array}\right)$$
        $$D^2=\left(\begin{array}{rr}
        
        169.5200  &  0.1400\\
        0.1200 & 240.6400
        \end{array}\right)$$
        $$D^2=\left(\begin{array}{rr}
        162.3200 &   0.1400\\
        0.1200&  232.9400
        \end{array}\right)$$
        $$p_1=\alpha r_1+\beta e_1 + \gamma s_1$$
        $$p_2=\alpha r_2+\beta e_2 + \gamma s_2$$
        $$p_{11}=p_1$$
        $$p_{12}=p_1-\gamma s_1^2$$
        $$p_{21}=p_2-\gamma s_2^2$$
        $$p_{22}=p_2$$
        $$P=\left(\begin{array}{cc}
        p_{11} & p_{12}\\
        p_{21} & p_{22}
        \end{array}\right)$$
        $$I=\{A,B\}$$
        $$S_1={a_1,a_2}$$
        $$S_2={b_1,b_2}$$
        $$C^1=\{c_{ij}^1\}_{2\times 2}$$
        $$C^2=\{c_{ij}^2\}_{2\times 2}$$
        $$G=\{S1,S2,C^1,C^2\}$$
        $$U_{d_1}=Yd_{11}+(1-Y)d_{21}$$
        $$U_{d_2}=Yd_{12}+(1-Y)d_{22}$$
        $$R_d=XU_{d_1}+(1-X)U_{d_2}$$
        $$F(X)=\dfrac{dX}{dt}=0$$
        $$U_{p_1}=Xp_{11}+(1-X)p_{21}$$
        $$U_{p_2}=Xp_{12}+(1-X)p_{22}$$
        $$R_p=YU_{p_1}+(1-Y)U_{p_2}$$
        $$F(Y)=\dfrac{dY}{dt}=0$$
        $$\xi_1=150$$
        $$\xi_2=244$$
        $$c_1=10$$
        $$c_2=12$$
        $$\eta_1=1.6$$
        $$\eta_2=1.8$$
        $$e_1=22$$
        $$e_2=17$$
        $$\eta_1=1.4$$
        $$\eta_1=1.5$$
        $$e_1=21$$
        $$e_2=19$$
        $$\eta_1=1.2$$
        $$\eta_2=1.3$$
        $$e_1=18.5$$
        $$e_2=16$$
        $$P_1=\left(\begin{array}{cc}
        0.2557 &   0.1796\\
        0.2189  &  0.2972
        \end{array}\right)$$
        $$P_2=\left(\begin{array}{cc}
        0.2586  &  0.1824\\
        0.1989  &  0.2772
        \end{array}\right)$$
        $$P_3=\left(\begin{array}{cc}
        0.2600 &   0.1839\\
        0.2332  &  0.3115
        \end{array}\right)$$
        \begin{align*}
        U_{d_1}&=Yd_{11}+(1-Y)d_{21}\\
        &=175.32Y+0.12(1-Y)
        \end{align*}
        \begin{align*}
        R_d&=XU_{d_1}+(1-X)U_{d_2}\\
        &=175.32XY+0.12X(1-Y)+0.14(1-X)Y+242.74(1-X)(1-Y)
        \end{align*}
        \begin{align*}
        F(X)&=\dfrac{dX}{dt}=X(U_{d_1}-R_d)\\
        &= 175.32XY+0.12X(1-Y)-175.32X^2Y-0.12X^2(1-Y)-0.14(1-X)XY-242.74(1-X)X(1-Y)
        \end{align*}
        $$F(X)=\dfrac{dX}{dt}=0$$
        \begin{align*}
        U_{p_1}&=Xp_{11}+(1-X)p_{21}\\
        &=0.2557X+0.2189(1-X)
        \end{align*}
        \begin{align*}
        R_p&=YU_{p_1}+(1-Y)U_{p_2}\\
        &=0.2557XY+0.2189(1-X)Y+ 0.1796X(1-Y)+0.2972(1-X)(1-Y)
        \end{align*}
        \begin{align*}
        F(Y)&=\dfrac{dY}{dt}=Y(U_{p_1}-Rp)\\
        &=0.2557XY+0.2189(1-X)Y-0.2557XY+0.2189(1-X)Y- 0.1796XY-0.2972(1-X)Y-0.2557XY^-0.2189(1-X)Y^2- 0.1796X(1-Y)Y-0.2972(1-X)(1-Y)Y
        \end{align*}
        $$F(Y)=\dfrac{dY}{dt}=0$$
        
        //
        \begin{align*}
        U_{d_1}&=Yd_{11}+(1-Y)d_{21}\\
        &=175.32Y+0.12(1-Y)
        \end{align*}
        \begin{align*}
        R_d&=XU_{d_1}+(1-X)U_{d_2}\\
        &=169.52XY+0.12X(1-Y)+0.14(1-X)Y+240.6(1-X)(1-Y)
        \end{align*}
        \begin{align*}
        F(X)&=\dfrac{dX}{dt}=X(U_{d_1}-R_d)\\
        &= 169.52XY+0.12X(1-Y)-169.52X^2Y-0.12X^2(1-Y)-0.14(1-X)XY-240.6(1-X)X(1-Y)
        \end{align*}
        $$F(X)=\dfrac{dX}{dt}=0$$
        \begin{align*}
        U_{p_1}&=Xp_{11}+(1-X)p_{21}\\
        &=0.2586X+0.1989(1-X)
        \end{align*}
        \begin{align*}
        R_p&=YU_{p_1}+(1-Y)U_{p_2}\\
        &=0.2557XY+0.1989(1-X)Y+ 0.1824X(1-Y)+0.2272(1-X)(1-Y)
        \end{align*}
        \begin{align*}
        F(Y)&=\dfrac{dY}{dt}=Y(U_{p_1}-Rp)\\
        &=0.2557XY+0.1989(1-X)Y-0.2557XY+0.1989(1-X)Y- 0.1824XY-0.2272(1-X)Y-0.2557XY^-0.1989(1-X)Y^2- 0.1824X(1-Y)Y-0.2272(1-X)(1-Y)Y
        \end{align*}
        $$F(Y)=\dfrac{dY}{dt}=0$$
        
        //
        \begin{align*}
        U_{d_1}&=Yd_{11}+(1-Y)d_{21}\\
        &=169.32Y+0.12(1-Y)
        \end{align*}
        \begin{align*}
        R_d&=XU_{d_1}+(1-X)U_{d_2}\\
        &=169.32XY+0.12X(1-Y)+0.14(1-X)Y+232.94(1-X)(1-Y)
        \end{align*}
        \begin{align*}
        F(X)&=\dfrac{dX}{dt}=X(U_{d_1}-R_d)\\
        &= 169.32XY+0.12X(1-Y)-169.32X^2Y-0.12X^2(1-Y)-0.14(1-X)XY-232.94(1-X)X(1-Y)
        \end{align*}
        $$F(X)=\dfrac{dX}{dt}=0$$
        \begin{align*}
        U_{p_1}&=Xp_{11}+(1-X)p_{21}\\
        &=0.2600X+0.2332(1-X)
        \end{align*}
        \begin{align*}
        R_p&=YU_{p_1}+(1-Y)U_{p_2}\\
        &=0.2600XY+0.2332(1-X)Y+ 0.1839X(1-Y)+0.3115(1-X)(1-Y)
        \end{align*}
        \begin{align*}
        F(Y)&=\dfrac{dY}{dt}=Y(U_{p_1}-Rp)\\
        &=0.2600XY+0.2332(1-X)Y-0.2600XY+0.2332(1-X)Y- 0.1839XY-0.3115(1-X)Y-0.2600XY^-0.2332(1-X)Y^2- 0.1839X(1-Y)Y-0.3115(1-X)(1-Y)Y
        \end{align*}
        $$F(Y)=\dfrac{dY}{dt}=0$$
    \end{appendices}
\end{document}