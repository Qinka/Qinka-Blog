\documentclass{ctexart}
\usepackage{listings}

\author{Qinka}
\title{数模大概}

\begin{document}
  \maketitle
  %
\section{About}
\label{sec:about}
这篇是一些列 post 的开始。
这些 post 是对数模对一些总结。里面会有一些各个方面的总结。

第一篇内容是关于 Matlab 中使用的一些基本套路。
其中的内容,并不一定是原创。
\endinput

  \section{Prime}
  \label{sec:prime}
  这一部分是关于 Matlab 中基本的操作。

  \paragraph{矩阵的操作}
  \label{par:mat:base}
  对于矩阵
  $$A=\left(\begin{array}
  a_{11} & a_{12} & \cdots & a_{1j} \\
  a_{21} & a_{22} & \cdots & a_{2j} \\
  \vdots & \vdots & \ddots & \vdots \\
  a_{i1} & a_{i2} & \cdots & a_{ij}
  \end{array}\right)$$

  如果想要访问或者“取出” 第 $p$ 行,第 $q$ 列的元素时需要
  \begin{lstlisting}[language=matlab]
    A(p,q)
  \end{lstlisting}

  如果是访问或者“取出” 第 $p$ 行的元素时需要
  \begin{lstlisting}[language=matlab]
    A(p,:)
  \end{lstlisting}

  如果是访问或者“取出” 第 $q$ 行的元素时需要
  \begin{lstlisting}[language=matlab]
    A(:,q)
  \end{lstlisting}

  以上的访问,还可以对其赋值
  \begin{lstlisting}[language=matlab]
    A(:,q) = zeros(1,q)
  \end{lstlisting}

  定义新的符号
  \begin{lstlisting}[language=matlab]
    sym x
  \end{lstlisting}

  产生新的$i$行$j$列的零矩阵
  \begin{lstlisting}[language=matlab]
    zeros(i,j)
  \end{lstlisting}
  如果是产生$i$阶的零矩阵
  \begin{lstlisting}[language=matlab]
    zeros(i)
  \end{lstlisting}

  产生新的$i$阶单位
  \begin{lstlisting}[language=matlab]
    eye(i)
  \end{lstlisting}
  如果产生$i$行$j$列的矩阵,多余行或者列填充0
  \begin{lstlisting}[language=matlab]
    eye(i,j)
  \end{lstlisting}

  生成$i$行$j$列全$1$的矩阵
  \begin{lstlisting}[language=matlab]
    ones(i,j)
  \end{lstlisting}
  生成$i$阶的全$1$的矩阵
  \begin{lstlisting}[language=matlab]
    ones(i)
  \end{lstlisting}

  产生随机数,范围在$[0,1]$之间
  \begin{lstlisting}[language=matlab]
    rand
    rand()
  \end{lstlisting}
  产生$i$阶随机数矩阵
  \begin{lstlisting}[language=matlab]
    rand(i)
  \end{lstlisting}
  产生$i$行$j$列的随机数矩阵
  \begin{lstlisting}[language=matlab]
    rand(i,j)
  \end{lstlisting}

  产生正态分布的随机数,服从 $x~N(0,1)$的正态分布
  使用命令 \lstinline|randn| 。

  产生线性等分向量
  \begin{lstlisting}[language=matlab]
    linspace(a,b)
    linspace(a,b,n)
  \end{lstlisting}
  其表示的是将 $[a,b]$ 或者 $[b,a]$ 的区域等分为$n$份,其中$n$的默认值为$100$。

  产生对数等分向量
  \begin{lstlisting}[language=matlab]
    logspace(a,b)
    logspace(a,b,n)
  \end{lstlisting}
  其表示

  产生汉克尔矩阵 (Hankel Matrix)
  $$H_n=\left(\begin{array}
  a_1    & a_2     & a_3      & \cdots & a_n     \\
  a_2    & a_3     & a_4      & \cdots & a_{n+1} \\
  a_3    & a_5     & a_5      & \cdots & a_{n+2} \\
  \vdots & \vdots  & \vdots   & \ddots & \vdots  \\
  a_n    & a_{n+1} & a_{n+2}  & \cdots & a_{2n}
  \end{array}\right)$$
  \begin{lstlisting}[language=matlab]
  \end{lstlisting}

  产生 希尔伯特(Hilbert)矩阵
  $$H_n=\left(\begin{array}
  1           & \frac{1}{2}   & \frac{1}{3}   & \cdots & \frac{1}{n}   \\
  \frac{1}{2} & \frac{1}{3}   & \frac{1}{4}   & \cdots & \frac{1}{n+1} \\
  \frac{1}{3} & \frac{1}{4}   & \frac{1}{5}   & \cdots & \frac{1}{n+2} \\
  \vdots      & \vdots        & \vdots        & \ddots & \vdots        \\
  \frac{1}{n} & \frac{1}{n+1} & \frac{1}{n+2} & \cdots & \frac{1}{2n}
  \end{array}\right)$$
  \begin{lstlisting}[language=matlab]
  \end{lstlisting}

  生成希尔伯特逆矩阵
  \begin{lstlisting}[language=matlab]
  \end{lstlisting}

  生成帕斯卡(Pascal)矩阵
  \begin{lstlisting}[language=matlab]
  \end{lstlisting}



  \paragraph{矩阵的基本属性}
  \label{par:mat:bc}

  矩阵的 特征值与特征向量的计算
  \begin{lstlisting}[language=matlab]
    [V,D] = eig(A)
  \end{lstlisting}
  其中 $V$ 是求解得到的 特征向量组成的矩阵, $D$ 是求解得到的特征值组成的矩阵。


\end{document}
