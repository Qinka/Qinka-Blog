\documentclass{ctexart}
\usepackage{listings}

\author{Qinka}
\title{数模大概}

\begin{document}
  \maketitle
  %
\section{About}
\label{sec:about}
这篇是一些列 post 的开始。
这些 post 是对数模对一些总结。里面会有一些各个方面的总结。

第一篇内容是关于 Matlab 中使用的一些基本套路。
其中的内容,并不一定是原创。
\endinput

  \section{Prime}
  \label{sec:prime}
  这一部分是关于 Matlab 中基本的操作。

  \subsection{矩阵的操作}
  \label{ssec:mat:base}
  对于矩阵
  $$A=\left(\begin{array}{cccc}
  a_{11} & a_{12} & \cdots & a_{1j} \\
  a_{21} & a_{22} & \cdots & a_{2j} \\
  \vdots & \vdots & \ddots & \vdots \\
  a_{i1} & a_{i2} & \cdots & a_{ij}
  \end{array}\right)$$

  如果想要访问或者“取出” 第 $p$ 行,第 $q$ 列的元素时需要
  \begin{lstlisting}[language=matlab]
    A(p,q)
  \end{lstlisting}

  如果是访问或者“取出” 第 $p$ 行的元素时需要
  \begin{lstlisting}[language=matlab]
    A(p,:)
  \end{lstlisting}

  如果是访问或者“取出” 第 $q$ 行的元素时需要
  \begin{lstlisting}[language=matlab]
    A(:,q)
  \end{lstlisting}

  以上的访问,还可以对其赋值
  \begin{lstlisting}[language=matlab]
    A(:,q) = zeros(1,q)
  \end{lstlisting}

  定义新的符号
  \begin{lstlisting}[language=matlab]
    sym x
  \end{lstlisting}

  产生新的$i$行$j$列的零矩阵
  \begin{lstlisting}[language=matlab]
    zeros(i,j)
  \end{lstlisting}
  如果是产生$i$阶的零矩阵
  \begin{lstlisting}[language=matlab]
    zeros(i)
  \end{lstlisting}

  产生新的$i$阶单位
  \begin{lstlisting}[language=matlab]
    eye(i)
  \end{lstlisting}
  如果产生$i$行$j$列的矩阵,多余行或者列填充0
  \begin{lstlisting}[language=matlab]
    eye(i,j)
  \end{lstlisting}

  生成$i$行$j$列全$1$的矩阵
  \begin{lstlisting}[language=matlab]
    ones(i,j)
  \end{lstlisting}
  生成$i$阶的全$1$的矩阵
  \begin{lstlisting}[language=matlab]
    ones(i)
  \end{lstlisting}

  产生随机数,范围在$[0,1]$之间
  \begin{lstlisting}[language=matlab]
    rand
    rand()
  \end{lstlisting}
  产生$i$阶随机数矩阵
  \begin{lstlisting}[language=matlab]
    rand(i)
  \end{lstlisting}
  产生$i$行$j$列的随机数矩阵
  \begin{lstlisting}[language=matlab]
    rand(i,j)
  \end{lstlisting}

  产生正态分布的随机数,服从 $x~N(0,1)$的正态分布
  使用命令 \lstinline|randn| 。

  产生线性等分向量
  \begin{lstlisting}[language=matlab]
    linspace(a,b)
    linspace(a,b,n)
  \end{lstlisting}
  其表示的是将 $[a,b]$ 或者 $[b,a]$ 的区域等分为$n$份,其中$n$的默认值为$100$。

  产生对数等分向量
  \begin{lstlisting}[language=matlab]
    logspace(a,b)
    logspace(a,b,n)
  \end{lstlisting}
  其表示与线性等分类似,其是将$10^i$中的$i$划分成线性空间。

  产生一输入$(x_1,x_2,\dots,x_n,\dots)$为对角线的矩阵
  $$\left(\begin{array}{ccccc}
  x_1    & 0      & \cdots & 0      & \cdots \\
  0      & x_2    & \cdots & 0      & \cdots \\
  \vdots & \vdots & \ddots & \vdots & \cdots \\
  0      & 0      & 0      & x_n    & \cdots \\
  \vdots & \vdots & \vdots & \vdots & \ddots
  \end{array}\right)$$
  \begin{lstlisting}[language=matlab]
  blkdiag(x1,x2,...,xn,...)
  \end{lstlisting}
  产生汉克尔矩阵 (Hankel Matrix)
  $$H_n=\left(\begin{array}{ccccc}
  a_1    & a_2     & a_3      & \cdots & a_n     \\
  a_2    & a_3     & a_4      & \cdots & a_{n+1} \\
  a_3    & a_5     & a_5      & \cdots & a_{n+2} \\
  \vdots & \vdots  & \vdots   & \ddots & \vdots  \\
  a_n    & a_{n+1} & a_{n+2}  & \cdots & a_{2n}
  \end{array}\right)$$
  \begin{lstlisting}[language=matlab]
  hankel(A1,A2)
  \end{lstlisting}
  其中
  \begin{eqnarray*}
  A_1 &=& (a_1,a_2,\cdots,a_i) \\
  A_2 &=& (a_{i+1},a_{i+2},\cdots,a_{2n})
  \end{eqnarray*}

  产生 希尔伯特(Hilbert)矩阵
  $$H_n=\left(\begin{array}{ccccc}
  1           & \frac{1}{2}   & \frac{1}{3}   & \cdots & \frac{1}{n}   \\
  \frac{1}{2} & \frac{1}{3}   & \frac{1}{4}   & \cdots & \frac{1}{n+1} \\
  \frac{1}{3} & \frac{1}{4}   & \frac{1}{5}   & \cdots & \frac{1}{n+2} \\
  \vdots      & \vdots        & \vdots        & \ddots & \vdots        \\
  \frac{1}{n} & \frac{1}{n+1} & \frac{1}{n+2} & \cdots & \frac{1}{2n}
  \end{array}\right)$$
  \begin{lstlisting}[language=matlab]
  hilb(n)
  \end{lstlisting}

  生成希尔伯特逆矩阵
  \begin{lstlisting}[language=matlab]
  invhilb(n)
  \end{lstlisting}

  生成帕斯卡(Pascal)矩阵,与杨辉三角一样的矩阵
  \begin{lstlisting}[language=matlab]
  pascal(n)
  pascal(n,0)
  \end{lstlisting}
  生成 Pascal 矩阵的下三角分量
  \begin{lstlisting}[language=matlab]
  pascal(n,1)
  \end{lstlisting}
  生成一个排位变换后的 \lstinline[language=matlab]|pascal(n,1)| 的版本。
  起返回结果是单位矩阵的多维数据集的根。
  \begin{lstlisting}[language=matlab]
  pascal(n,2)
  \end{lstlisting}

  生成托普利矩阵
  $$\left(\begin{array}{cccccc}
  a_1     & b_{m-n+2} & b_{m-n+3} & \cdots & b_{m-1} & b_m \\
  a_2     & a_1     & b_{m-n+2} & \cdots & b_{m-2} & b_{m-1}\\
  a_3     & a_2     & a_1     & \cdots & b_{m-3} & b_{m-2}\\
  \vdots  & \vdots  & \vdots  & \ddots & \vdots  & \vdots \\
  a_{n-1} & a_{n-2} & a_{n-3} & \cdots & a_1     & b_{m-n+2} \\
  a_n     & a_{n-1} & a_{n-2} & \cdots & a_2     & a_1
  \end{array}\right)$$
  \begin{lstlisting}[language=matlab]
  toeplitz(a,b)
  \end{lstlisting}
  如果 $\forall i \in [0,n) ,\, b_{m-i}=a_{n-i}$
  则
  \begin{lstlisting}[language=matlab]
  toeplitz(a)
  \end{lstlisting}
  可生成该矩阵。

  \subsection{基本运算}
  \label{ssec:mat:bc}

  矩阵的 特征值与特征向量的计算
  \begin{lstlisting}[language=matlab]
  [V,D] = eig(A)
  \end{lstlisting}
  其中 $V$ 是求解得到的 特征向量组成的矩阵, $D$ 是求解得到的特征值组成的矩阵。

  向量的点乘运算
  $$A\cdots B$$
  \begin{lstlisting}[language=matlab]
  dot(A,B)
  \end{lstlisting}
  如果输入的 $A$ 与$B$是矩阵,则对应的列向量做点积,形成一个新的向量。

  向量的叉乘
  $$A \times B$$
  \begin{lstlisting}[language=matlab]
  cross(A,B)
  \end{lstlisting}

  两向量,必须为严格的向量经行卷积
  $$A*B=C$$
  \begin{lstlisting}[language=matlab]
  conv(A,B)
  \end{lstlisting}
  反之,反卷积为
  \begin{lstlisting}[language=matlab]
  deconv(A,B)
  \end{lstlisting}
 
  求集合的交集,集合 $A$ 与 集合 $B$ 表示成矩阵后
  \begin{lstlisting}[language=matlab]
  intersect(A,B)
  \end{lstlisting}
  
  求解包含关系,对于矩阵表示的集合 $A$ 来说,同样以矩阵表示的集合 $B$ 中的
  每一个元素 $b_i$,如果 $b_i \in A$ 则记为逻辑 $1$,反之为零
  \begin{lstlisting}[language=matlab]
  ismember(B,A)
  \end{lstlisting}
  用这个命令,将得到对应集合$B$中元素的一个逻辑矩阵。
  
  计算两个用矩阵表示的集合的差
  $$A-B$$
 \begin{lstlisting}[language=matlab]
 setdiff(A,B)
 \end{lstlisting}
 
 计算两个用矩阵表示的集合的交集的非,也就是异或,换句话说,就是把两个集合中独有的元素取出然后组成一个新的集合
 \begin{lstlisting}[language=matlab]
 setxor(A,B)
 \end{lstlisting}
 
 求并集
 \begin{lstlisting}[language=matlab]
 union(A,B)
 \end{lstlisting}
 
 取集合的单值元素
 \begin{lstlisting}[language=matlab]
 unique(A)
 \end{lstlisting}
 
 矩阵的除法。对于运算
 $$C=AB$$
 其中$A$,$B$,与$C$ 均是矩阵,合乎乘法的要求。
 则如果由$C$与$B$计算$A$则
 \begin{lstlisting}[language=matlab]
 C/B
 \end{lstlisting}
 如果想通过计算得到$B$则
 \begin{lstlisting}[language=matlab]
 A\C
 \end{lstlisting}
 其中约束, $A$ 与$B$ 均为可逆矩阵。
 
 
 
 \begin{lstlisting}[language=matlab]
 
 \end{lstlisting}
\end{document}
