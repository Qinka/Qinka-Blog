\documentclass{ctexart}
\author{李约瀚 \\John Lee \\ 14130140331 \\ qinka@live.com \\ qinka@qinka.pw}
\title{技术壁垒到底挡住了什么}

\usepackage{cite}
\usepackage{hyperref}

\begin{document}
    \maketitle
    
    现在实际上有许许多多的壁垒充斥的我们的世界。贸易壁垒与技术壁垒是两类常见的壁垒。
    并不是所有的壁垒都是好的。壁垒是用来保护的,但是有时则会适得其反。
    这里只会讨论计算机方面的技术壁垒,应为在计算机方面有许多但其石头砸自己脚的例子。
    
    \section{争丢的商标}
    
    上世纪,计算机操作系统领域曾今发生过的 关于 Unix 商标的争夺。以至于 Windows NT 对于对 Unix 的兼容,
    都望而却步。AT\&T 与贝尔实验室 在搭建技术壁垒的时候,却断送了 Unix 与 Unix 的未来。
    
    \subsection{兼容有罪}
    
    在微软开发 Windows NT 的时候,微软的目标是工作站与服务器。做为 Unix 的传统领地。
    如果 微软开发的 Windows NT 与 Unix 是兼容的,或者是在“形式”上兼容,以便方便的将代码移植。
    而且 当时 BSD 针对 Unix 开发的,让 Unix 真正丰满的套件 是开放源代码的。
    尽管我们不能确定,当时微软是否有过想过利用 BSD 套件,但是毋庸置疑的是,如果 Windows NT 
    当初是兼容 Unix,或者是 Unix 类的系统。当下操作系统的世界则将不可能预测。
    
    兼容有罪,在美国的司法历史上不止依次,此外还有 wine 项目中,兼容 Windows 那些东西,就曾经遭到
    调查。
    
    \subsection{Unix 初始}
    Unix 诞生在贝尔实验室的一台 PDP-7 小型机上。在这样的一个年代,计算机“公司”获利的方式基本上是硬件。
    软件共享的概率还是比较大的。应为限制发展的因素很多,应用程序并没有现在这样容易移植。
    Unix 的世界中,一方面 System X 系列做为商业化的产品销售,但是处于共享精神,提供与各大大学研究与使用。
    
    从贝尔实验室 与 AT\&T 在BSD 和 其余大学早期侵犯 版权销售系统时,睁一只眼闭一只眼的情况来看。包括 BSD
    在内的 美国各大高校对 Unix 或者成 System X 的贡献度相当不小。以至于 所谓的真正的版权所有者,宁愿放弃
    追究的权利,也不愿意失去美国各大高校对 Unix 体系的贡献
    \footnote{事实上,Unix的贡献者不局限于美国\cite{taoup.{2012}.{cn}}。}。
    
    而出于各种各样的目的, 贝尔实验室 等没有这只各种各样的壁垒,当然也没有意识到 Unix 真正的商业价值。
    当然真正没有遇见的盈利模式时依赖于服务而非代码本身。
    
    在 Unix 诞生指出,大家对于添砖加瓦的行为,并没有太多的特别的想法。
    并不会纠结于,代码版权的问题。甚至于 Unix 本身则是“三个人头脑风暴了一把”,然后把代码写了出来
    \cite{taoup.{2012}.{cn}}然而,世事难料,没人知道 “几天之后的” Unix 将编程世人所争抢的。
    
    \subsection{Unix 的商业化}
    在 1970年代, Unix 甚至到了随处可见的地步,然而没有什么人在意版权问题\cite{unix-history}。
    但是谁会向导今天,当“盗用”了某个 公司的专有软件,却会被比开发团队还要强大的法务团队,满世界追的跑。
    当然这基本是开玩笑的。但是当人们几乎不在乎这些的时候,贝尔实验室这样的地方开始发掘出来 Unix 潜在商业
    价值。当人们认为 AT\&T在反托拉斯的诉讼下,Unix 会变的向以往一样,开放的时候,人们却发现,有更多的人
    开始发觉 Unix 的价值。
    
    正如淘金热一样,从事 Unix 商业而掘金的公司,变的过了起来。而也如淘金热一般,大家都想将 Unix 商业化,然后能赚的一桶金子,
    并且将金矿占为己有 —— 使自己的的 Unix 的遍独特化。但是谁也没有料到 Unix 的商业化之后, Unix 的大地变的千疮百孔。
    
    就像一群小脚老太太站在庭院门口吵架一样, Unix 的商业化带来利益的同时,也带来了争端。一方面版权问题成了家常便饭。
    另一方面, 各种各样的 Unix 系统对于使用者来说,带来这高昂的价格,与几乎不同用的问题。商业到眼红的 Unix 厂商
    以至于在 2000年之后 还是对着 Linux 系统“内部的”东西宣称着版权,并起诉这。以至于微软在决定支付版权费用,破财消灾之时,
    还会被告上法庭。以要求微软将版权费用交付给真正的版权所有者——原告。
    
    争来争去,Unix不光整丢 Unix 系统的市场份额,连 大型机甚至小型机的市场份额都跟着一起丢失。
    
    \subsection{Unix 的对手}
    在 Unix 大肆商业化的时候,高校坐不住了。 以前说好的 提供给 (美国)高校研究的系统,成了需要高价购买的产品。
    当然,肯定不会有人妥协,于是有了 Minix 的诞生。一个提供教学目的的 Unix 类的免费操作系统。麻雀虽小,五脏俱全。
    这个“里程碑”似的系统加上“里程碑”似的的 Linus,发酵出来了 Linux。

    另一边 基于 Intel 的 80x86 CPU 系列的个人计算机与 微软的 Windows 系列操作系统的迅速占领市场。当 Unix 的世界的老派黑客
    \footnote{不得不提到的一点是:在中国与美国,媒体带着极其富有偏见的心态,将本来是毫无贬义之一的 黑客(hacker)一词 与  骇客(crasher)混为一谈。
    黑客 与 hacker 一词本身指的是,一群有着不烦的创造力的、对计算机只是掌握较深,有着极强动手的计算机研究人员与开发者,而非对利用非法手段,非法对他人计算机进行各类非法行为的人。
    后者那种非法行为的人是被称作 骇客或者 crasher。}
    还在对 8086 这样低劣的处理器与 摩托罗拉 68000 处理器的高雅 评头论足的时候。
    那些似乎会被 Unix 世界黑客所嘲笑的,不了解计算机的,甚至连图形界面都玩不转的群体,却在 低廉的 PC 机与 PC 兼容机 和 Windows 系统的“帮助下”
    开始了现代的计算机之旅,并迅速成为了使用计算机群体中的一大部分人。
    这还不算完,老派黑客们在还没有意识到到底发生了什么的时候,老派黑客引以为傲的 Unix 与 大/小型机,却不再是主流。
    Linux 与 Windows NT 取代了 Unix 之前的大部分市场,Intel 的 80x86 系列的处理器 在发展过程中 派生出种类丰富的、性能优异的但是极其廉价的 微型处理器。
    雨后春笋般出现的小型公司,采用了 以 Intel 的 x86 架构处理器为核心的微型计算机,并搭配中新生而富有活力的 Linux 操作系统,或者强大的 Windows NT 系统。
    众多依赖 Windows 的商业公司,在 Windows 与 Intel 达成某种交易之后,这个市场的份额几乎被 Windows 与 Intel 处理器所垄断。

    \subsection{争端}
    在对 Unix 所有权争论不休,以至于 新型的 Linux 系统与 微软公司 频繁的中枪。甚至于到了明明知道应该缴纳版权费用,却有着数不清的人伸出手来并说着他才是真正的
    合法的收钱的人。当然, Linux 中枪也是意料之中的事情,总是有所谓的 版权所有人来控告 Linux 内部的某个东西侵犯版权。
    尽管 Windows 与 Unix 系统外表来说千差万别,微软还是被要求缴纳某些版权费用。最后,缴纳版权费的微软还是被控告,“原因”是把钱缴纳给了错误的一方。

    Unix 内部的新派黑客,也不断的嚷嚷着、呻吟着、哀嚎着 Unix 的末日即将到来。许多 Unix 中有着“名人”,转向 GNU 与 Linux 一方,继续着曾经的梦想。
    
    \subsection{尾声}
    在 Unix 的商标 最后的所有者成为 OpenGroup, 操作系统端 Linux 与 Windows 几乎瓜分天下。而大型机的市场,在 廉价的 PC 机的冲击下,变得没有什么优势。
    而 正如 Linus 所说:“Software is like sex -- it's better when it's free.” 而且可能 包括在 Unix 老派黑客在内的很多人,根本没有
    想到如今 微软会在 GitHub 有一个组织账号\href{https://github.com/microsoft}{github.com/Microsoft} 然后里面会有着众多来自 Windows
    平台,开源、高效,同时被移植到各种 平台的 工具,软件,框架\footnote{其中比较重要的例子(笔者认为)有
      \href{https://github.com/microsoft/powershell}{PowerShell},\href{https://github.com/microsoft/vscode}{visual studio code},
      与最重要的\href{https://github.com/microsoft/??}{.Net Core}}。
    
    令人惋惜的是, Unix 世界的“震荡”带来了其他的诸多的问题,比如降低了世界的软件开发过程中的质量,但是也为像 Linux 这样的东西提供了机会。
    或许老话说的好:旧的不去,新的不来。永恒不变的只有变化。

    \section{哲学}
    万物终将归结于哲学?我不做评论。但是方法论是关键的。上古的哲学家们\footnote{请原谅我这样用词,半开玩笑的。}总结出来的一些哲学,其中
    很多都是用来规避壁垒的。比如 Linus 说过的:“Software is like sex -- it's better when it's free.”\footnote{这句话的大致的含义是:软件应该想性
    那样自由。虽然笔者对此有些不同的想法——自由应该是法式的那种来自美的自由。但是不得不说,总结的真不错。} 这句话阻挡掉了许多壁垒,其中包含着 让
    Unix 奄奄一息的那种壁垒。

    不得不说,Linus 这这句话一语中的的指出了计算机软件尤其是基础软件,比如操作系统,在计算机普及之后,(从某种意义上说),计算机软件如同
    性。我们抛开道德层面,做为对人权的尊重,性这样的有些敏感的内容本身来说,算是人权最基本的东西。在现在来看,计算机软件,尤其是基础软件
    对于一个计算机来说,对于一个人来说,在某种意义上是一种基本权利,正如吃饭睡觉一样。假设当睡觉这样的行为存在版权一样的限制的时候,
    以至于说我们睡觉的时候,还要按照时间缴纳类似于版权费一样的东西时,这样人类的世界是无法想象的。
    当有人质疑的时候,我只能说,我们需要奉献精神。
    
    \subsection{壁垒一:文档}
    有一句老话说得好\footnote{可能这句话并没有太长的年限,但是相对软件行业高速的变化,虽然没有前端社区那样——明天的已经是过时的,但是还是相对还是,
        十年前,已经算是远古了。}
    :六个月后的你将不再是你。
    

    \begin{thebibliography}{0}
        \bibitem{taoup.{2012}.{cn}}《Unix编程艺术》,【美】 Eric\space S.Raymond 著,姜宏,何源,蔡晓俊\space 译,电子工业出版社,2012,北京。
        \bibitem{unix-history} History\space and \space Timeline, The\space Open\space GROUP,
        \href{http://www.unix.org/what_is_unix/history_timeline.html}{www.unix.org/what\_is\_unix/history\_timeline}.
    \end{thebibliography}
        
\end{document}