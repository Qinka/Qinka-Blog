\documentclass{ctexart}
\author{李约瀚 \\John Lee \\ 14130140331 \\ qinka@live.com \\ qinka@qinka.pw}
\title{技术壁垒到底挡住了什么}

\usepackage{cite}
\usepackage{hyperref}

\begin{document}
    \maketitle
    
    现在实际上有许许多多的壁垒充斥的我们的世界。贸易壁垒与技术壁垒是两类常见的壁垒。
    并不是所有的壁垒都是好的。壁垒是用来保护的,但是有时则会适得其反。
    这里只会讨论计算机方面的技术壁垒,应为在计算机方面有许多但其石头砸自己脚的例子。
    
    \section{争丢的商标}
    
    上世纪,计算机操作系统领域曾今发生过的 关于 Unix 商标的争夺。以至于 Windows NT 对于对 Unix 的兼容,
    都望而却步。AT\&T 与贝尔实验室 在搭建技术壁垒的时候,却断送了 Unix 与 Unix 的未来。
    
    \subsection{兼容有罪}
    
    在微软开发 Windows NT 的时候,微软的目标是工作站与服务器。做为 Unix 的传统领地。
    如果 微软开发的 Windows NT 与 Unix 是兼容的,或者是在“形式”上兼容,以便方便的将代码移植。
    而且 当时 BSD 针对 Unix 开发的,让 Unix 真正丰满的套件 是开放源代码的。
    尽管我们不能确定,当时微软是否有过想过利用 BSD 套件,但是毋庸置疑的是,如果 Windows NT 
    当初是兼容 Unix,或者是 Unix 类的系统。当下操作系统的世界则将不可能预测。
    
    兼容有罪,在美国的司法历史上不止依次,此外还有 wine 项目中,兼容 Windows 那些东西,就曾经遭到
    调查。
    
    \subsection{Unix 初始}
    Unix 诞生在贝尔实验室的一台 PDP-7 小型机上。在这样的一个年代,计算机“公司”获利的方式基本上是硬件。
    软件共享的概率还是比较大的。应为限制发展的因素很多,应用程序并没有现在这样容易移植。
    Unix 的世界中,一方面 System X 系列做为商业化的产品销售,但是处于共享精神,提供与各大大学研究与使用。
    
    从贝尔实验室 与 AT\&T 在BSD 和 其余大学早期侵犯 版权销售系统时,睁一只眼闭一只眼的情况来看。包括 BSD
    在内的 美国各大高校对 Unix 或者成 System X 的贡献度相当不小。以至于 所谓的真正的版权所有者,宁愿放弃
    追究的权利,也不愿意失去美国各大高校对 Unix 体系的贡献
    \footnote{事实上,Unix的贡献者不局限于美国\cite{taoup.{2012}.{cn}}。}。
    
    而出于各种各样的目的, 贝尔实验室 等没有这只各种各样的壁垒,当然也没有意识到 Unix 真正的商业价值。
    当然真正没有遇见的盈利模式时依赖于服务而非代码本身。
    
    在 Unix 诞生指出,大家对于添砖加瓦的行为,并没有太多的特别的想法。
    并不会纠结于,代码版权的问题。甚至于 Unix 本身则是“三个人头脑风暴了一把”,然后把代码写了出来
    \cite{taoup.{2012}.{cn}}然而,世事难料,没人知道 “几天之后的” Unix 将编程世人所争抢的。
    
    \subsection{Unix 的商业化}
    在 1970年代, Unix 甚至到了随处可见的地步,然而没有什么人在意版权问题\cite{unix-history}。
    但是谁会向导今天,当“盗用”了某个 公司的专有软件,却会被比开发团队还要强大的法务团队,满世界追的跑。
    当然这基本是开玩笑的。但是当人们几乎不在乎这些的时候,贝尔实验室这样的地方开始发掘出来 Unix 潜在商业
    价值。当人们认为 AT\&T在反托拉斯的诉讼下,Unix 会变的向以往一样,开放的时候,人们却发现,有更多的人
    开始发觉 Unix 的价值。
    
    正如淘金热一样,从事 Unix 商业而掘金的公司,变的过了起来。而也如淘金热一般,大家都想将 Unix 商业化,然后能赚的一桶金子,
    并且将金矿占为己有 —— 使自己的的 Unix 的遍独特化。但是谁也没有料到 Unix 的商业化之后, Unix 的大地变的千疮百孔。
    
    就像一群小脚老太太站在庭院门口吵架一样, Unix 的商业化带来利益的同时,也带来了争端。一方面版权问题成了家常便饭。
    另一方面, 各种各样的 Unix 系统对于使用者来说,带来这高昂的价格,与几乎不同用的问题。商业到眼红的 Unix 厂商
    以至于在 2000年之后 还是对着 Linux 系统“内部的”东西宣称着版权,并起诉这。以至于微软在决定支付版权费用,破财消灾之时,
    还会被告上法庭。以要求微软将版权费用交付给真正的版权所有者——原告。
    
    争来争去,Unix不光整丢 Unix 系统的市场份额,连 大型机甚至小型机的市场份额都跟着一起丢失。
    
    \subsection{Unix 的对手}
    在 Unix 大肆商业化的时候,高校坐不住了。 以前说好的 提供给 (美国)高校研究的系统,成了需要高价购买的产品。
    当然,肯定不会有人妥协,于是有了 Minix 的诞生。一个提供教学目的的 Unix 类的免费操作系统。麻雀虽小,五脏俱全。
    这个“里程碑”似的系统加上“里程碑”似的的 Linus,发酵出来了 Linux。

    \begin{thebibliography}{0}
        \bibitem{taoup.{2012}.{cn}}《Unix编程艺术》,【美】 Eric\space S.Raymond 著,姜宏,何源,蔡晓俊\space 译,电子工业出版社,2012,北京。
        \bibitem{unix-history} History\space and \space Timeline, The\space Open\space GROUP,
        \href{http://www.unix.org/what_is_unix/history_timeline.html}{www.unix.org/what\_is\_unix/history\_timeline}.
    \end{thebibliography}
        
\end{document}