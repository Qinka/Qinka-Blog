\documentclass{ctexart}

\author{Qinka//qinka@live.com//qinka@qinka.pw}
\title{Read Me}

\usepackage{hyperref}

\begin{document}
\maketitle

\section{废话}

我总是相信版权会消失,尤其是在软件方面。并不是说它真的会消失,而是相对不重要。我一直对这样的唐突的想法没有什么系统的理解。
直到看到了《Unix 编程艺术》,版权“消失”是显而易见的。对于开源软件,版权这一事物相对与协议变得更加不重要了。
版权与协议也只是成了一个限制“使用者”的套路,人们明白开源的好处,直到开源能带来的价值,设计一个套路让未明白开源意义的文明人先于遵守。

如果说要让不明白开源意义的人,深刻理解开源带来什么,出门直走,E.S.Raymond 的《大教堂与集市》。
初中同座常说的一句话:“没文化,真可怕。”不理解开源的人,对于软件开发来说,则是灾难性的。比如说一个当了一辈子商人的人去当总统,没有一点从政经验,
尽管是资本主义制度下的长老,除非立法限制大枣制成的药丸,否则这个国家一定会吃。一个不深刻理解开源的国家,不深刻理解计算机的国家,
宛如一个大老粗去当文联主席,让一个利益熏心的资本家管理一个支部。

\subsection{差即是好}

差即是好,开源中有一种谦虚。一方面是无论自己做的真正有多好,谦逊才是重点。而另一方面,真正有勇气“不要脸”让大家评论自己的代码的人,才会收回。

\subsection{姿势}

开源的姿势很简单,代码开放,常发布,合作。这是《大教堂与集市》中所“提倡”的三个规则。
然而事实上规则有很多,繁而杂。成为一门“宗教”、“哲学”。正如我们的诗歌,诗词中平仄、押韵,以至于“推”、“敲”二字的典故早就当代汉语中“推敲”一词。


\subsection{正题}
做什么都得做的有模有样。对于一个开源的项目,我们通常会有一些“习惯法”,对于这样的一个开源的行为,黑客们通常有一些“习惯”。
E.S.Raymond 在 《Unix 编程艺术》 一书中有一些介绍,但是在现在——“时代不同了”,有一些我观察到的新的一些习惯的产生。
\section{命名风格}

提倡 GNU 的命名风格,是因为它真的不错。通常 GNU 的版本号的风格是 \verb|主干+major.minor.patch| 的方式,而且通常只包含小写字母数字。
通常又这样的顺序:
\begin{enumerate}
\item 项目名称
\item 横杠
\item 版本号
\item 点
\item “src” 或者是 “bin”
\item 点号(横线也可)
\item 二进制类型或者选项
\item 归档扩展名
\end{enumerate}

E.S.Raymond 在 《Unix 编程艺术》 中举了几个例子。我就把它们在复制出来,此外再从 GNU 官方拷贝一些例子。

\subsection{错误的例子}
\begin{itemize}
	\item  \verb|foobar123.tar.gz| E.S.Raymond 说像没有版本号的“foobar123”,在我看来,也许是 12年3月份的一个版本,正如同 Ubuntu 的命名风格。
		当然称之为1月23号的版本,也不为过。
	\item \verb|foobar-v4.1.tar.gz| GitHub 官方在 Tag/Release 这方面的建议 tag 的格式大概像 \verb|v1.2-rc.3|,但是此处的字符 v,
		更像项目名称中的一个东西。
	\item \verb|foo_bar-4.1.tar.gz| 一般下划线会被认作是空格的替代,而一般这类名称中 不存在真正的空格。
\end{itemize}


\subsection{正确的}
\begin{itemize}
	\item \verb|freetalk-4.1.tar.gz| GNU 的一个软件。正确的命名风格告诉我们,这个叫 freetalk 的软件,版本为 4.1。下载的归档则是源代码。
	\item \verb|yesod-core-1.4.31.tar.gz|  Hackage 上 Haskell 中一个基于 wai 的 Web 应用框架。这个归档是源代码
\end{itemize}

\subsection{压缩格式}
在 E.S.Raymond 的书中,在 GitHub 的提供的下载的源文件中,还有一些像 GNU 在内的传统的自由软件“智造商”在内,通常大家习惯使用 tar 归档,
并使用 GNU ZIP 压缩。通常的归档工具 tar 都提供直接解压并解档 \verb|tar.gz| 后缀名的压缩归档的功能,当然也有直接归档并压缩的功能。
而现在则是提倡使用 xz 的压缩方式,压缩效率比 GNU ZIP 格式的要高不少。现在的 tar 归档工具列提供类似的对于 xz 的支持。
目前 Linux 内核提供的代码的格式就是 \verb|tar.xz| 格式的。

\section{自动化工具}

% auto*
% CI
% CD


\section{归档压缩}
在传统的自由软件阵地,人们往往愿意使用 GNU ZIP 的压缩方式并搭配上 tar 的归档方式。文件扩展名通常是 \verb|tar.gz| ,也有少数文件的
扩展名是 \verb|tgz|。相对的,现在很多地方开始使用 xz 作为压缩工具,比如说 Linux 的内核代码。
xz 与 Windows 平台上免费而高效率的 7z 相对来说是近亲使,都用到了 LZMA 压缩压缩算法。 xz 相对 gz 来说,压缩效率相对来说较高,
但是相对的 gz 的压缩的速度要比 xz 要快不少。

此外,不推荐使用一些闭源,比如 RAR,因为这些软件会牵扯到版权的问题,同时大部分自由软件的开发者并不使用这一类软件。

\section{README 与 GitHub}
\label{readmeNgithub}
\subsection{Git}
Git 简单的来说是一种 分布式的版本控制系统。传统的源代码分发的方式是将代码归档压缩之后通过各种途径分发。
而现在很多源码分发的软件的则通过 Git。 通过直接使用 Git 克隆代码,用户可以方便的获取到代码。而同时作出贡献也变得容易许多。

开发过程中的协作方式有很多种,协作的软件也有很多。有时开玩笑的来说,协作可以通过 U 盘,也可以通过邮件或者通信软件(比如QQ)。
而真正高效而正确的方式是通过 Git。
\subsection{Markdown}

\section{创建文件}
通常,创建的开源的项目或者工具在文件命名上是有一定的习惯“法”的。黑客们在这方面有不少习惯。
通常会有将源代码放入 \verb|src| 文件夹的习惯,但是这并不是最重要的。下面的一些内容是有关于一些“开门见山”的文件。
\paragraph{README}
这个文件通常是被最先阅读的。相当于指路标。更详细的内容在 \ref{readmeNgithub} 说明。
\paragraph{INSTALL}
这个文件往往是使用最为关心的文件之一,在使用之前,与配置、编译、安装,也就是人们常说的“三部曲”,相关的内容,都应该在这个文档中指明。
\paragraph{AUTHORS}
无论我们在不在乎版权,项目的贡献者都应该罗列出来以表示感谢。同时也有使用 \verb|MAINTAINERS| 作为文件名的,类似的文件还有 \verb|CREDITS|。
\paragraph{NEWS}
这个是有关项目的最新的内容,新闻。但是者事实上更适合与写在“自媒体”上。
\paragraph{HISTORY}
这个文件与项目的历史更为相关但是,相对使用 Git/GitHub 的项目,这一记录更多的应该其索引的作用。
\paragraph{CHANGE}
重要的改动应该写到这里面。
\paragraph{COPYING}
根据 E.S.Raymond 的 《Unix 编程艺术》,这是 GNU 项目的惯例。在中文译本中标注的“项目许可证条款”。
但是在 很多地方更提倡使用 \verb|LICENSE| 这个名字来指代许可证。
\paragraph{LICENSE}
项目许可证条款。GitHub 等平台默认的使用这个文件作为协议。
\paragraph{FAQ}
这也是一个比较重要的文件。FAQ 是 Frequently Asked Questions 的缩写。这个文件应该用于回答常见问题,可减轻维护者的负担。


\section{邮件列表与网站}
\subsection{邮件列表}
在 E.S.Raymond 的《Unix 编程艺术》书中 ESR 提到对于邮件列表这样的东西。


\section{许可证}



\end{document}