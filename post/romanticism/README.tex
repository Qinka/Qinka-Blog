\documentclass{ctexart}

\author{Qinka//qinka@live.com//qinka@qinka.pw}
\title{Read Me}

\usepackage{hyperref}

\begin{document}
\maketitle

\section{废话}

我总是相信版权会消失,尤其是在软件方面。并不是说它真的会消失,而是相对不重要。我一直对这样的唐突的想法没有什么系统的理解。
直到看到了《Unix 编程艺术》,版权“消失”是显而易见的。对于开源软件,版权这一事物相对与协议变得更加不重要了。
版权与协议也只是成了一个限制“使用者”的套路,人们明白开源的好处,直到开源能带来的价值,设计一个套路让未明白开源意义的文明人先于遵守。

如果说要让不明白开源意义的人,深刻理解开源带来什么,出门直走,E.S.Raymond 的《大教堂与集市》。
初中同座常说的一句话:“没文化,真可怕。”不理解开源的人,对于软件开发来说,则是灾难性的。比如说一个当了一辈子商人的人去当总统,没有一点从政经验,
尽管是资本主义制度下的长老,除非立法限制大枣制成的药丸,否则这个国家一定会吃。一个不深刻理解开源的国家,不深刻理解计算机的国家,
宛如一个大老粗去当文联主席,让一个利益熏心的资本家管理一个支部。

\subsection{差即是好}

差即是好,开源中有一种谦虚。一方面是无论自己做的真正有多好,谦逊才是重点。而另一方面,真正有勇气“不要脸”让大家评论自己的代码的人,才会收回。

\subsection{姿势}

开源的姿势很简单,代码开放,常发布,合作。这是《大教堂与集市》中所“提倡”的三个规则。
然而事实上规则有很多,繁而杂。成为一门“宗教”、“哲学”。正如我们的诗歌,诗词中平仄、押韵,以至于“推”、“敲”二字的典故早就当代汉语中“推敲”一词。

\section{命名风格}

提倡 GNU 的命名风格,是因为它真的不错。通常 GNU 的版本号的风格是 \verb|主干+major.minor.patch| 的方式,而且通常只包含小写字母数字。
通常又这样的顺序:
\begin{enumerate}
\item 项目名称
\item 横杠
\item 版本号
\item 点
\item “src” 或者是 “bin”
\item 点号(横线也可)
\item 二进制类型或者选项
\item 归档扩展名
\end{enumerate}

E.S.Raymond 在 《Unix 编程艺术》 中举了几个例子。我就把它们在复制出来,此外再从 GNU 官方拷贝一些例子。

\subsection{错误的例子}

\verb|foobar123.tar.gz| E.S.Raymond 说像没有版本号的“foobar123”,在我看来,也许是 12年3月份的一个版本,正如同 Ubuntu 的命名风格。当然称之为1月23号的版本,也不为过。

\subsection{正确的}

\verb|freetalk-4.1.tar.gz| GNU 的一个软件。正确的命名风格告诉我们,这个叫 freetalk 的软件,版本为 4.1。下载的归档则是源代码。


\section{自动化工具}

% auto*
% CI
% CD

\section{制作实践}

\subsection{归档压缩}

\subsection{README}

\subsection{创建文件}

\subsection{邮件列表与网站}

\subsection{许可证}



\end{document}