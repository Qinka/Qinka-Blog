

\documentclass{ctexart}


\author{Qinka \\ qinka@live.com \\ qinka@qinka.pw}
\title{Euterpea..}


\begin{document}

\section{Euterpea}

Euterpea 是 耶鲁大学计算机系的一个课程,目的是讲述 音乐(信号)与函数式编程。将二者综合在一起讲解,(他们似乎认为)更有效果。
同时 \href{http://www.euterpea.com}{Euterpea} 也是一个 Haskell 的库。这个库与这一门课——\href{http://haskell.cs.yale.edu/euterpea/}{Euterpea}
是有 耶鲁大学计算机系的 \href{http://haskell.cs.yale.edu}{The Yale Haskell Group} 负责,其下设了
\href{http://haskell.cs.yale.edu/euterpea/euterpea-studio/}{Euterpea Studio} ,是主要的“活动场地”。

\section{My Delirium}

《My Delirium》一歌是由 Pascal Gabriel, Hannah Robinson, Phillipa Brown, 与 Alex Gray 作词曲,并由 Ladyhawke 演唱的歌曲,在 2007年。

对从 Sheet Music Plus 网站购买到琴谱\footnote{从谱上标注的和弦的指位来看,应该是吉他谱,这里强行当钢琴谱了。} ,然后作死。


\section{DEMO}

由于版权问题,可能无法将完整的代码与音乐拿出来。

\paragraph{常规软件制作版本}
使用常规软件制作的版本的连接:\href{/res/music/my-delirium-qinka.mp3}{my-delirium-qinka.mp3}
使用常规软件制作的大体完成的版本的链接:\href{/res/music/my-delirium-qinka-2017-03-18-1.mp3}{my-delirium-qinka-2017-03-18-1.mp3}
使用常规软件制作的大体完成的合成乐器乱入版的链接:\textit{忘记生成文件了,远程编辑这个博客所以无法上传}
使用常规软件制作的一个铃声的链接:\href{/res/music/my-delirium-qinka-ring-1.mp3}{my-delirium-qinka-ring-1.mp3}


\end{document}
