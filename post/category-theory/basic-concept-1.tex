\documentclass{ctexart}
\author{Qinka<qinka@live.com>}
\title{基础概念-1 \\ Basic Concept 1}
\usepackage{mathrsfs}

\begin{document}
\section{集合与类}
\subsection{集合}
G.Cantor 的想法是由一些确定的而且相互区别的对象汇聚成的一个整体,称之为集合\cite{ACT{2012}},相对那些对象就是集合中的元素。
而这些元素的组合方式有两个原则\cite{ACT{2012}}:
\begin{description}
\item[外延原则] 集合是由元素确定的。对于两个元素完全相同的集合,我们说他们是同一集合。
\item[概括原则] 对于一个性质 $P$ 与一个集合$S$,我们可以这样表示:
$$ S = \{ x | P(x) \} $$
\end{description}

\paragraph{罗素悖论}
1902 年 罗素发现,当 $P(x) = x \notin x $ 构成的集合 $T$ :
$$T = \{ x | x \notin x \}$$
明显$T$ 不是集合。
\subsection{类}
由于罗素悖论,集合的定义显然是不正确的。于是我们使用“类”这一概念,替换其中的集合。集合一定是类,但是类不一定是集合。
其中那些不是集合的类,我们称之为\textbf{真类}。然后我们重新定义以取代前先的集合的概括定理。
\begin{description}
\item[外延原则] 一个类由其元素唯一确定。
\item[概括原则] 对于任意一个性质 $P$ ,总是对应一个类 $S$ 恰好能表示为如下:
  $$ S = \{ x | P(x) \} $$
\item[原则 1] 对于不含任一元素的集合 $\varnothing$,称之为空集。
\item[原则 2] 对于任意两个集合 $X$ 与 $Y$,均存在一个集合 $S$,其元素恰好是 $X$ 与 $Y$。
  称 $S$ 为无序偶集合。当 $X = Y$ 时,简写为 $\{X\}$ 并称之为 $X$ 的单元素集合。

  若令 $$<X,Y> = \{\{X\},\{X,Y\}\}$$,则称 $<X,Y>$ 为 $X$ 与 $Y$ 的有序偶集合。
\item[原则 3] 对于任意集合 $S$ 存在 $S$ 的幂集 $ \mathscr{P}(X)=\{ X| X \subseteq S \} $。
\end{description}

\begin{thebibliography}{0}
\bibitem{ACT{2012}} 王兵山,毛晓光,刘万伟.高级范畴论.北京:清华出版社,2012.
\end{thebibliography}
\end{document}