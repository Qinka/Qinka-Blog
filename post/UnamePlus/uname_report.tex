\documentclass{ctexart}

\author{李约瀚(Qinka) \\ 14130140331 \\ qinka@live.com \\ qinka@qinka.pw}
\title{Uname Language Report}

\begin{document}
\maketitle

\section{概述与版权声明}
\label{sec:copyright}

这个是绘图语言 Uname 的语言报告,基本上是在模仿 Haskell 的语言报告。
这个报告的作者为 李约瀚 \footnote{西安电子科技大学,软件学院,14级本科生。qinka@live.com 与 qinka@qinka.pw 均为本人的邮箱。}

Uname 这个语言是根据2016年 西安电子科技大学软件学院编译原理课程中要求实现的绘图语言,经过可以忽略的修改而来。
语言“原型”相关权利归西安电子科技大学软件学院编译原理授课组与相关人士所有。 Uname 语言的相关内容的所有权归本人(李约瀚)所有。

这个语言及其派生的(如 UnamePlus)的相关内容将遵循类 BSD3 协议的开源协议。但特此在此说明,冒用他人的东西验收学校相关课程及实验属于对
所有权与署名的侵犯。

\section{前言}
\label{sec:preface}

在 西安电子科技大学软件学院 不知到多久以前,编译原理的授课组搞出来一个连名字都没有的绘图语言。
估计数由于太简单了吧,整个语言的语句也没有几个,十分适合本科生学习编译的一些基础知识。
在2016年的这门课程上,我“决定”实例化这个语言原型,并将之命名为 Uname。并且尝试拓展这个语言,来更深入的学习编译原理。
由于没有名字,那就叫 Uname 了。

\subsection{目标}
\label{sec:preface:goals}

\begin{enumerate}
\item 遵循课程中的语言所涉及到的规范
\item 使得语言所需要的解释器变得相对十分简单
\item 语言简洁、稳定
\end{enumerate}

\subsection{编译器}
\label{sec:preface:compiler}

关于编译器,在编写这个报告的时候,还没有开始制作编译器。

\subsection{语言扩展}
\label{sec:preface:extension}

关于语言扩展,将在专门的一节讨论,其中语言扩展的目标与目的是在一定程度上扩展语言,达到 UnamePlus 的预期。
换句话说 UnamePlus 将通过 Uname 的语法扩展实现。而每一类扩展,将组成一个范畴。

\section{导入}
\label{sec:intro}

Uname 是一个及其简单的绘图语言,真的简单的不能再简单了。Uname 的绘图方式也是十分简单的,直接是按像素点绘制的。
给出绘制的参数方程与参数的范围、步长,来画出图形。

这个报告尽可能不与具体的编译器/解释器实现相挂钩。但不能排除会涉及到部分编译器的内容。

\subsection{程序结构}
\label{sec:intro:structure}

Uname 的程序结构相当简单,但是简单的东西往往能作出复杂的事情。原生的 Uname 程序分为两个部分,一个是调整坐标,
也就是坐标的变换。另一个部分就是绘制图形。

就滨湖岸坐标而言,变换只有三个变换,也是最基本的变换:平移,伸缩,与旋转。
绘制图形也是比较简单,一个简单的参数方程。通过简单的迭代,就可以完成,以至于编译器/解释器的实现中,最复杂的可能成为
绘制图形这一部分。

\subsection{语言集合}
\label{sec:intro:langset}

Uname 的语言内容集合,不是简单的一个集合。 Uname/Uname+ 相对来说有一个本质上的集合,暂且我们称之为 Uname Normal Kernel(UNK)。
Uname Normal Kernel 并不是一个最小的完备集合。原生的 Uname 才是一个最小的完备的、满足课程要求的集合。
我们将原生的 Uname 的语言内容组成的集合称之为 Uname Mini Kernel(UMK)。同时我们对于扩展的部分,也就是 UnamePlus,
称之为 Uname Plus Kernel (UPK)。而 Uname 整个语言是有这三个部分组成的。当然,并不是简单的并集关系。
所以这个语言就变成了一个类,而非集合,根据罗素悖论。

具体对于语言集合的内容,将在语言与语法扩展部分,详细说明。

\subsection{语言内容}
\label{sec:intro:lang}

语言的实体部分包括四个部分:坐标变换的设置、绘图、注释、与常用函数。
坐标的三种变换方式将通过“IS”语句说明。绘图则是那唯一的语言。注释则稍有不同,对于 UMK,只有 C-like 的与 Haskell
的 行注释格式。对于 UNK 与 UPK 将有块注释。



\section{语言与语法扩展}
\label{sec:languageextension}

\subsection{类与集合}
\label{sec:langexte:classNset}

这一部分是关于语言与语法扩展的关系。对于 $Uname$ (简记为 $U$) 类中的三个元素: UMK,UNK,与 UPK,相对的定一个类关于语法扩展:$Ext$。
然后对于这个内容,存在一系列的二元关系组成的集合 $ R \subseteq Uname \times Ext $。 这个二元关系是使得 UMK、UNK、与UPK
三者之间不完全的转换。举个例子,对于 注释的语法扩展,会使得 UNK 或 UPK 只能使用 UMK 中的语法格式,但不会是其他部分也与UMK 一样。
但是做一种变换,即认为,讨论某种语法扩展对于语言处于哪种集合的时候,只考虑很小的一部分源代码,也就是该扩展其作用的地方。
使得 我们的转换的二元关系的是完整的,也就是说“语言状态”由 UxK 转换成了UyK。

取一个语法扩展的子集 $E_u \subseteq Ext$ 与对应的 $R_u \subseteq R$ , 并定义两个映射 $source$ 与 $target$ ,均为
$ R_u \rightarrow U$ 其中含义是 二元关系(每个关系会对应一个语法扩展) 映射到 对应的原核集合 与变化核集合。
则 四元组 $<U,R_u,source,target>$ 组成了图,我们称之为语法扩展图。

我们定义一组特殊的二元关系 $R_\varepsilon$,对于 $ \forall x,y \in U$ 当对于某一语法或者语法扩展,$x$ 与 $y$ 是一样的,比如注释,
UNK 与 UPK 的注释是一样的,而与 UMK 不是一样的,则由二元关系 $r$ 满足从 x 到 由的转化,或者 $\widetilde{r}$ 满足 从 y 到 x 的转化。

则,对于一类语法扩展,额外再定义一个复合运算映射,则使得对于一类语法扩展组成的图与复合运算组成的 $<U,R_u,dom,cod,\circ>$ 五元组
形成一个范畴。则 $Ext$ 集合对应的语法扩展组成的一个范畴的集合。

% 包含两方面,扩展语言,与使得语言接近 编译原理原生的那个语言。也就是超集与子集。

\end{document}