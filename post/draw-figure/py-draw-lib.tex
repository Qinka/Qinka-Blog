
\documentclass{ctexart}

\author{Qinka \\ qinka@live.com \\ qinka@qinka.pw}
\title{Python 绘图的 “中断”指南}

\usepackage{listings}


\begin{document}
%
\maketitle
%
%
{\tiny “中断” 指南的含义是大致与 Unix 编程艺术中提及的 manpage (等 传统手册)中“需要通篇阅读而理解手册”所相反。
  希望能达到,几乎不需要太多上下相关内容,即可对部分内容使用并理解。}
\newpage

%% matplotlib
\section{matplotlib}

\paragraph{介绍}
matplotlib 是 一个 python 的 2D 的绘图库 其主页是 \href{http://matplotlib.org}{matplotlib.org}。

\subsection{快速画图}

\paragraph{简单划线}
这个将在坐标中画出 一条直线,从点$(1,1)$ 到 $(4,4)$。
\begin{lstlisting}[language=python]
plot([1,2,3,4],[1,2,3,4])
\end{lstlisting}

\paragraph{创建图}
在创建一个 \textit{figure},其中 \lstinline|figure=(x,y)| 会设置图片的大小为 $8 \times 6$ 的大小,\lstinline|dpi=k| 会设置 图片的 \verb|dpi|。
\begin{lstlisting}[language=python]
figure(figure=(10,9),dpi=300)
\end{lstlisting}

\paragraph{创建子图}
如果需要创建一个子图的话使用 \lstinline|subplot| 命令创建并定位。
\begin{lstlisting}[language=python]
subplot(row,col,position)
\end{lstlisting}
其中 \verb|col| 是指子图有多少列, \verb|row| 是指子图有多少行,\verb|position| 是将当前画图的“句柄”定位到第\verb|position|个子图上\footnote{一行一行的数。}

\paragraph{坐标轴限制}
当需要限制坐标轴的时候,可使用坐标轴的限制工具来限制。
\begin{lstlisting}[language=python]
xlim(-1.0,3.0)
ylim(1,10)
\end{lstlisting}
将x坐标限制在$-1$到$3$之间,将y坐标限制在$1$与$10$之间。

\paragraph{坐标轴刻度}
设置坐标的刻度。
\begin{lstlisting}[language=python]
xticks([1,2,3])
yticks([1,2,3])
\end{lstlisting}

\paragraph{显示}
\begin{lstlisting}[language=python]
show()
\end{lstlisting}


\subsection{参考}
\href{http://liam0205.me/2014/09/11/matplotlib-tutorial-zh-cn/}{Matplotlib 教程}
\href{http://blog.csdn.net/ywjun0919/article/details/8692018}{Python图表绘制:matplotlib绘图库入门}


\section{Seaborn}
\subsection{参考}
\href{https://segmentfault.com/a/1190000005092460}{【译】Seaborn教程(一) 管理图表的艺术}
\href{http://python.jobbole.com/80853}{Python和数据科学的起步指南}

\section{ggplot}
\subsection{参考}
\href{https://www.plob.org/article/7264.html}{《使用ggplot2画图》}

\section{Bokeh}
\subsection{参考}
\href{http://youerning.blog.51cto.com/10513771/1709113}{Bokeh快速入门(1)}

\section{pygal}
\subsection{参考}
\href{http://rfyiamcool.blog.51cto.com/1030776/1378400}{python使用pygal进行绘制数据图表和监控图表}

\section{Plotly}
\subsection{参考}
\href{http://www.xueqing.tv/cms/article/93}{用plotly包创建交互式网页图形}


\section{geoplotlib}
\subsection{参考}
\href{https://github.com/andrea-cuttone/geoplotlib}{github:andrea-cuttone/geoplotlib}

\section{Gleam}
\subsection{参考}
\href{https://github.com/dgrtwo/gleam}{github:dgrtwo/gleam}


\end{document}