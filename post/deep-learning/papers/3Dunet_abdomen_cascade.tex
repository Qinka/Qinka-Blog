\section{摘要}
    \paragraph{主要贡献}
        在避免使用人工构造的特征与训练特殊目的的模型。同时在全新的数据集中D_{score} 能达到 $68.5\%$ 到 $82,2\%$。
    \paragraph{特色方法}
        使用 3D U-Net (3D FCN) 对 voxel 表示的 CT 数据进行处理。使用两阶段的 FCN 网络逐步圈定候选范围,然后是较为精准的分割结果。
    \paragraph{主要内容}
        使用 3D FCN 网络 3D U-Net构造两阶段的器官分割网络。在第一阶段中,利用 U-Net 大致分割出各个器官的候选位置,然后利用第二阶段筛选出
        更加准确的候选范围。作者使用 331 份临床数据训练网络,然后再全新的,从未被使用的 150 份来自其他医院的临床数据进行测试,其中
        D_{score} 能达到 $68.5\%$ 到 $82,2\%$。此外该网络与2D FCN 进行了对比,然后并测试了鲁棒性。

        
